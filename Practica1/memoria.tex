\documentclass{article}
\usepackage{graphicx}
\usepackage{listings}
\usepackage{courier}
\usepackage{xcolor}
\usepackage{float}
\definecolor{dkgreen}{rgb}{0,0.6,0}
\definecolor{mauve}{rgb}{0.58, 0, 0.82}
\lstset{
  basicstyle=\small\ttfamily, % Configura la fuente básica del código
  breaklines=true, % Permite que las líneas de código se rompan si son demasiado largas para la página
  frame=single, % Añade un marco alrededor del código
  keepspaces=true, % Mantiene los espacios en el código, útil para mantener la indentación
  showstringspaces=false, % Evita que los espacios en las cadenas de texto se subrayen
  numbers=left, % Muestra los números de línea a la izquierda
  numberstyle=\tiny\color{gray}, % Configura el estilo de los números de línea
  keywordstyle=\color{blue}, % Configura el color de las palabras clave
  commentstyle=\color{dkgreen}, % Configura el color de los comentarios
  stringstyle=\color{mauve}, % Configura el color de las cadenas de texto
}
\title{Memoria de la práctica 1 de Desarollo de Software}
\author{Ana Isabel Mena Meseguer \\ Jose Luis Parra Azor\\ Elena Vallejo Ruiz \\ Alejandro Ruiz Salazar}
\date{}
\begin{document}
\maketitle
\section{Ejercicio 1}
Para la práctica, partimos de este diagrama UML, que a su vez hemos ido completando conforme íbamos creando las clases y las íbamos uniendo. 
\begin{figure}[H]
\centering
\includegraphics[width=\textwidth]{C:/Users/PC/Dropbox/DS/P1/img/Ejercicio1.drawio.png}
\caption{UML usado para los ejercicio 1 y 2}
\end{figure}
FactoriaCarreraYBicicleta es una interfaz, de las que heredan FactoriaCarrera y FactoriaMontaña, y Carrera y Bicicleta son clases abstractas de las que heredan CarreraMontaña y CarreraCarretera, y BicicletaMontaña y BicicletaCarretera respectivamente. 
Nuestro main pide un número por pantalla para crear los ArrayList de bicicletas tanto de montaña como de carretera y crea las carreras correspondientes con los ArrayList. Luego llamamos a los métodos start de thread que invoca a los métodos run que hemos creado para CarreraMontaña y CarreraCarretera. 
\begin{figure}[H]
\centering
\includegraphics[width=\textwidth]{C:/Users/PC/Dropbox/DS/P1/img/ejecucionej1.jpeg}
\caption{Salida por pantalla del ejercicio 1}
\end{figure}
Estos métodos lo que hacen es retirar el porcentaje de bicicletas que corresponda con el método retirarBicis y luego muestra por pantalla las bicicletas que quedan con el método correr. Usando start lo que conseguimos es que los dos métodos run de CarreraMontaña y CarreraCarretera se ejecuten de manera concurrente (mediante hebras).
\begin{figure}[H]
\centering
\includegraphics[width=\textwidth]{C:/Users/PC/Dropbox/DS/P1/img/bicismon.jpeg}
\caption{Método run de CarreraMontaña}
\end{figure}
\begin{figure}[H]
\centering
\includegraphics[width=\textwidth]{C:/Users/PC/Dropbox/DS/P1/img/biciscar.jpeg}
\caption{Método run de CarreraCarretera}
\end{figure}
\section{Ejercicio 2}
Para el ejercicio 2 lo que hemos tenido que hacer ha sido, en esencia, lo mismo que para el ejercicio 1. Al no poder usar interfaces en python, las hemos sustituido por clases abstractas. También tenemos que tener cuidado ya que no podemos usar como tipo la clase abstracta padre (a diferencia de java, donde se podía usar como tipo para instanciar clases hija), por lo que llamamos directamente a la clase de la cual queremos instanciar nuestros objetos. Finalmente, para implementar el patrón de diseño prototipo, hemos clonado las bicicletas cambiando luego su identificador. 
\begin{lstlisting}[language=Python, caption={main.py}]
import copy
from FactoriaMontana import FactoriaMontana
from FactoriaCarretera import FactoriaCarretera

if __name__ == "__main__":
    factoriaMontana = FactoriaMontana()
    bicicletasMontana = []
    factoriaCarretera = FactoriaCarretera()
    bicicletasCarretera = []

    # Creacion de la bicicleta de montana original
    bicicleta = factoriaMontana.crear_bicicleta(1)
    bicicletasMontana.append(bicicleta)

    # Inclusion del resto de bicicletas a partir de una copia de la original
    for i in range(9):
        bicicleta_aux = copy.deepcopy(bicicleta)
        bicicleta_aux.set_id(i + 2)
        bicicletasMontana.append(bicicleta_aux)
    carreraMontana = factoriaMontana.crear_carrera("Montana", bicicletasMontana)

    # Creacion de la bicicleta de carretera original
    bicicleta = factoriaCarretera.crear_bicicleta(1)
    bicicletasCarretera.append(bicicleta)

    # Inclusion del resto de bicicletas a partir de una copia de la original
    for i in range(9):
        bicicleta_aux = copy.deepcopy(bicicleta)
        bicicleta_aux.set_id(i + 2)
        bicicletasCarretera.append(bicicleta_aux)
    carreraCarretera = factoriaCarretera.crear_carrera("Carretera", bicicletasCarretera)

    # Mostrar las bicicletas y las carreras
    print(carreraMontana.tipo())
    bicicletasMontana = carreraMontana.get_bicicletas()
    for bicicleta in bicicletasMontana:
        print(str(bicicleta.get_id()) + " " + bicicleta.tipo())

    print("\n" + carreraCarretera.tipo())
    bicicletasCarretera = carreraCarretera.get_bicicletas()
    for bicicleta in bicicletasCarretera:
        print(str(bicicleta.get_id()) + " " + bicicleta.tipo())   
\end{lstlisting}
\begin{figure}[H]
\centering
\includegraphics[width=0.9\textwidth]{C:/Users/PC/Dropbox/DS/P1/img/ej2output.png}
\caption{Salida por pantalla del ejercicio 2.}
\end{figure}
\section{Ejercicio 3}
Para este ejercicio hemos implementado el patrón composite dos veces. Nuestra idea trata sobre los elementos de una web, que se dividen en Página o Contenido, donde una pagina puede tener dentro mas páginas o contenido, y el contenido a su vez puede ser o texto o imágenes, donde un texto puede tener dentro mas textos o imágenes. 
\begin{figure}[H]
\centering
\includegraphics[width=\textwidth]{C:/Users/PC/Dropbox/DS/P1/img/umlej3.png}
\caption{UML del ejercicio 3}
\end{figure}
Para probar que estuviese bien implementado, el main que hemos hecho crea varios objetos de cada clase. Primero hacemos algunas inserciones correctas como añadir una página dentro de otra página o un contenido dentro de una pagina, y luego hemos hecho unas cuantas inserciones erróneas para mostrar por pantalla mensajes de error al intentar realizarlas, como por ejemplo meter un texto dentro de una imagen, una imagen dentro de otra imagen o una pagina dentro de un contenido.
\begin{figure}[H]
\centering
\includegraphics[width=\textwidth]{C:/Users/PC/Dropbox/DS/P1/img/ejecucionej3.png}
\caption{Salida por pantalla del ejercicio 3}
\end{figure}
\end{document}